\documentclass[a4paper,12pt]{report}
\usepackage[portuguese,brazilian]{babel}
\usepackage[utf8]{inputenc}
\usepackage[T1]{fontenc}
\usepackage{latexsym}
\title{Uma infraestrutura para o desenvolvimento de aplicações distribuídas baseada em minitransações}
\author{Leandro Ferro Luzia}
\date{2012}

\begin{document}
\pagestyle{headings}
\maketitle
\section*{Resumo}
O desenvolvimento de aplicações distribuídas envolve lidar, em geral, com a troca de mensagens na rede entre os computadores para que os dados relevantes à execução - o estado do sistema - possa ser compartilhado e conhecido por todos. Esta abordagem é complexa e requer o uso de protocolos para gerenciar e manter consistente entre todas as máquinas o estado compartilhado entre o sistema.

Uma forma comum de compartilhar o estado entre os componentes do sistema é utilizar algum tipo de armazenamento de dados como ponto de integração como, por exemplo, bancos de dados relacionais. A abordagem que utiliza um banco de dados relacional é interessante devido ao alto grau de controle que estes bancos oferecem, como integridade referencial e a execução de comandos que alteram diversos dados de forma atômica (transações), ou seja, ou todos os comandos são concluídos com êxito ou nenhum comando é executado. Esta utilização consistente dos dados é altamente desejável pois torna o desenvolvimento de aplicações mais simples de ser abordado, sem a necessidade de tratamentos complexos de controle de consistência por parte do desenvolvedor da aplicação. Porém, devido aos mecanismos para controlar a consistência dos dados, os bancos de dados relacionais passam a ser penalizados em relação à quantidade cada vez maior de dados que aplicações como comércio eletrônico ou redes sociais precisam armazenar. Essa grande quantidade de dados passa a exigir que o armazenamento seja particionado entre diferentes máquinas o que, por sua vez, torna o controle de consistência mais complexo. Dessa forma, a maneira usual com que os bancos de dados relacionais podem atender um número crescente de dados é através da melhora no hardware em que eles são executados, conhecida como escala de forma vertical. Enquanto é desejável que o sistema ofereça a consistência dos dados como um banco de dados relacional oferece, também se torna necessário que o sistema possa escalar de forma horizontal, permitindo particionar e armazenar os dados em máquinas distintas. 

Neste trabalho buscamos desenvolver uma infraestrutura que permite aplicações compartilharem estado de forma escalável horizontalmente e fazendo uso de minitransações para garantir a consistência dos dados em operações que envolvem a alteração de dados que estejam em diferentes máquinas. As minitransações são uma extensão do protocolo de transação distribuída de duas fases (Two-Phase Commit Protocol - 2PC) que permite acessar atomicamente ou modificar condicionalmente dados distribuídos em várias máquinas. Uma primitiva de programação é disponibilizada aos desenvolvedores de aplicação permitindo que passem a tratar o compartilhamento de estado entre o sistema como a criação e manipulação de estruturas de dados compartilhadas, excluindo a necessidade de lidar com protocolos e mecanismos de troca de mensagens explicitamente.
\tableofcontents
\listoftables
\listoffigures

\chapter{Introdução}
Com a popularização dos computadores, das redes de computadores e, de forma especial, da Internet, os sistemas computacionais tem sido expostos a um número cada vez maior de usuários - tanto humanos quanto computacionais. Desta forma diversos tipos de sistemas (como comércio eletrônico, motores de busca ou redes sociais, por exemplo) precisam suportar uma ampla variação, em termos de números de usuários atendidos e serviços oferecidos, quantidade de dados armazenados e manipulados, requerimentos em relação à taxa de processamento, número de elementos computaticionais que compõem o sistema, distribuição geográfica e tamanho das redes e dispositivos de armazenamento. 

Uma das maneiras de construir sistemas computacionais que atendam aos requisitos apresentados anteriormente é na forma de um sistema distribuído, que é um conjunto de elementos computacionais independentes que passa a impressão, para quem utiliza este sistema, de ser um único elemento computacional \cite{ds-tanenbaum}. Ao construir um sistema computacional de forma distribuída podemos aproveitar de forma mais eficiente os recursos computacionais disponíveis; a disponibilidade do sistema pode ser aumentada pois, com mais elementos computacionais disponíveis para execução, diminui a chance de que o sistema esteja indisponível; e o crescimento do sistema pode ser tratado de forma gradual e sobre demanda. Porém, devido à natureza distribuída destes sistemas problemas podem ser encontrados, que poderiam não existir caso o sistema fosse centralizado, como questões de segurança, falhas de comunicação devido a problemas nas redes de conexão e a chance de falhas, pois agora temos um número maior de componentes no sistema.

A capacidade de um sistema de operar de forma eficiente e com qualidade em relação à variação dos requisitos apresentados anteriormente é chamada de escalabilidade \cite{evaluating-scalability}. A escalabilidade de um sistema pode ser classificada de duas formas \cite{scale-up-vs-down}: escalabilidade {\em vertical} ou {\em horizontal}.

A maneira tradicional de escalar os sistemas era baseada na {\em escalabilidade vertical}, que se refere à capacidade de aumentar a eficiência da execução do sistema ao melhorar o hardware em que o sistema é executado, como por exemplo utilizar processadores com mais núcleos, maior quantidade de memória, discos mais rápidos e confiáveis, etc... Em geral, Sistemas Gerenciadores de Banco de Dados (SGBD's) são escalados utilizando essa abordagem.

Empresas com sistemas voltados à Internet (como Google, eBay, Amazon, Facebook) tem adotado a abordagem da {\em escalabilidade horizontal}, que se refere à capacidade de aumentar a eficiência da execução do sistema ao adicionar mais elementos computacionais ao sistema, permitindo distribuição de carga e aumento da disponibilidade do sistema.

\chapter{Descrição da infraestrutura}
Este trabalho pretende apresentar e construir uma infraestrutura para facilitar o desenvolvimento de sistemas distruídos escaláveis. Esta infraestrutura é baseada no trabalho desenvolvido em \cite{sinfonia} em que os autores apresentam {\em Sinfonia}, um sistema que cria um espaço linear de memória global distribuído entre diversos elementos de memória (memory nodes) e disponibiliza como primitiva de programação central uma {\em minitransação}, que permite acesso eficiente e consistente aos dados distribuídos entre os elementos de memória, enquanto que encapsula os complexos tratamentos em relação ao acesso concorrente e falhas. Desta forma, este sistema permite que o desenvolvimento de sistemas distribuídos seja vista do ponto de vista da modelagem e manipulação de estruturas de dados ao invés dos protocolos de comunicação entre os elementos do sistema. Esta primitiva de minitransação é utilizada pelas aplicações acessar de forma atômica e condicionalmente modificar dados dentro da infraestrutura.

Em \cite{sinfonia}, os autores assumem que o sistema distribuído executará no ambiente de uma central de dados (data center), um local que pode conter até milhares de máquinas, conectados através de redes cuja latência é baixa e nas quais particionamentos sejam raros de ocorrer. Estes pressupostos não são verdadeiros em outros ambientes, como redes de longa distância (Wide Area Network - WAN), sistemas ponto a ponto ou a Internet. Estes ambientes apresentam uma alta taxa de erros de comunicação, uma maior latência (junto com uma maior variação da mesma) de comunicação, e uma complexidade extra em relação à configuração e gerenciamento. A intenção da nossa infraestrutura é prover os serviços que {\em Sinfonia} oferece, porém em um ambiente mais genérico, como a Internet. 

A infraestrutura consistirá de elementos memória, que armazenam de fato os dados das aplicações, e uma biblioteca que executa nas máquinas das aplicações. Cada elemento de memória mantém uma sequência de palavras de memória, com contudo opaco ao sistema. Estas palavras são organizadas em um espaço de endereçamento linear, sem considerar nenhuma estrutura específica. O espaço de endereçamento de cada elemento de memória é separado, e o endereçamento completo de uma palavra é feito pelo par {\em (Identificador do nó, Endereço da palavra)}. Desta forma as aplicações podem aproveitar a semântica dos dados e utilizar um mesmo nó para armazenar informações relacionadas.

A primitiva de {\em minitransação} disponibilizada pela sistema permite que uma aplicação atualize dados em diferentes elementos de memória enquanto que oferece garantias relacionadas à atomicidade, consistência, isolamento e durabilidade. Esta primitiva é construída a partir de modificações ao protocolo de efetivação de duas fases (Two phase commit -2PC) utilizado no controle de transações distribuídas (TODO encontrar a referência correta para o 2PC - acho q eh do gray...). 

\chapter{Conclusão}

\begin{thebibliography}{99}

\bibitem{ds-tanenbaum} Tanenbaum, A.S., van Steen, M.: {\em Distributed Systems: Principles and Paradigms, 2nd Edition} (2006)

\bibitem{evaluating-scalability} Jogalekar, P., Woodside, M.: {\em Evaluating the scalability of distributed systems}. In IEEE Transactions on Parallel and Distributed Systems, Volume 11, Issue 6 (2000)

\bibitem{scale-up-vs-down} Michael, M., Moreira, J.E., Shiloach, D., Wisniewski, R.W.: {\em Scale-up x Scale-out: A Case Study using Nutch/Lucene}. In IEEE Parallel and Distributed Processing Symposium (2007)

\bibitem{sinfonia} Aguilera, M.K., Merchant, A., Shah, M., Veitch, A., Karamanolis, C.: {\em Sinfonia: a new paradigm for building scalable distributed systems}. Em SOSP, páginas 159-174 (2007)

\end{thebibliography}

\end{document}
