\documentclass[11pt,twoside,a4paper]{book}
%\usepackage[portuguese,brazilian]{babel}
\usepackage[utf8]{inputenc}
\usepackage[T1]{fontenc}
\usepackage{latexsym}

\usepackage[brazil,brazilian]{babel}
%\usepackage[latin1]{inputenc}
\usepackage[pdftex]{graphicx}           
\usepackage{setspace}                   
\usepackage{indentfirst}                
\usepackage{makeidx}                  
\usepackage[nottoc]{tocbibind}     
\usepackage{courier}                    
\usepackage{type1cm}              
\usepackage{listings}                   
\usepackage{titletoc}
%\usepackage[bf,small,compact]{titlesec} 
\usepackage[fixlanguage]{babelbib}
\usepackage[font=small,format=plain,labelfont=bf,up,textfont=it,up]{caption}
\usepackage[usenames,svgnames,dvipsnames]{xcolor}
\usepackage[a4paper,top=2.54cm,bottom=2.0cm,left=2.0cm,right=2.54cm]{geometry} % margens
%\usepackage[pdftex,plainpages=false,pdfpagelabels,pagebackref,colorlinks=true,citecolor=black,linkcolor=black,urlcolor=black,filecolor=black,bookmarksopen=true]{hyperref} % links em preto
\usepackage[pdftex,plainpages=false,pdfpagelabels,pagebackref,colorlinks=true,citecolor=DarkGreen,linkcolor=NavyBlue,urlcolor=DarkRed,filecolor=green,bookmarksopen=true]{hyperref} % links coloridos
\usepackage[all]{hypcap}                % soluciona o problema com o hyperref e capitulos
\usepackage[square,sort,nonamebreak,comma]{natbib}  
\fontsize{60}{62}\usefont{OT1}{cmr}{m}{n}{\selectfont}
\usepackage{fancyhdr}
\pagestyle{fancy}
\fancyhf{}
\renewcommand{\chaptermark}[1]{\markboth{\MakeUppercase{#1}}{}}
\renewcommand{\sectionmark}[1]{\markright{\MakeUppercase{#1}}{}}
\renewcommand{\headrulewidth}{0pt}
\graphicspath{{./figuras/}}             
\frenchspacing                          
\urlstyle{same}                         
\makeindex                              
\raggedbottom                           
\fontsize{60}{62}\usefont{OT1}{cmr}{m}{n}{\selectfont}
\cleardoublepage
\normalsize
% Ref: http://en.wikibooks.org/wiki/LaTeX/Packages/Listings
\lstset{ %
language=Java,                  % choose the language of the code
basicstyle=\footnotesize,       % the size of the fonts that are used for the code
numbers=left,                   % where to put the line-numbers
numberstyle=\footnotesize,      % the size of the fonts that are used for the line-numbers
stepnumber=1,                   % the step between two line-numbers. If it's 1 each line will be numbered
numbersep=5pt,                  % how far the line-numbers are from the code
showspaces=false,               % show spaces adding particular underscores
showstringspaces=false,         % underline spaces within strings
showtabs=false,                 % show tabs within strings adding particular underscores
frame=single,	                % adds a frame around the code
framerule=0.6pt,
tabsize=2,	                    % sets default tabsize to 2 spaces
captionpos=b,                   % sets the caption-position to bottom
breaklines=true,                % sets automatic line breaking
breakatwhitespace=false,        % sets if automatic breaks should only happen at whitespace
escapeinside={\%*}{*)},         % if you want to add a comment within your code
backgroundcolor=\color[rgb]{1.0,1.0,1.0}, % choose the background color.
rulecolor=\color[rgb]{0.8,0.8,0.8},
extendedchars=true,
xleftmargin=10pt,
xrightmargin=10pt,
framexleftmargin=10pt,
framexrightmargin=10pt
}

%----------------------------------------------
%\title{Uma infraestrutura para o desenvolvimento de aplicações distribuídas baseada em minitransações}
%\author{Leandro Ferro Luzia}
%\date{2012}
%-----------------------------------------------

% Corpo do texto
\begin{document}
\frontmatter 
\fancyhead[RO]{{\footnotesize\rightmark}\hspace{2em}\thepage}
\setcounter{tocdepth}{2}
\fancyhead[LE]{\thepage\hspace{2em}\footnotesize{\leftmark}}
\fancyhead[RE,LO]{}
\fancyhead[RO]{{\footnotesize\rightmark}\hspace{2em}\thepage}

\onehalfspacing

% ---------------------------------------------------------------------------- %
% CAPA
% Nota: O título para as dissertações/teses do IME-USP devem caber em um 
% orifício de 10,7cm de largura x 6,0cm de altura que há na capa fornecida pela SPG.
\thispagestyle{empty}
\begin{center}
    \vspace*{2.3cm}
    \textbf{\Large{Uma infraestrutura para desenvolvimento de aplicações distribuídas baseada em minitransações}}\\
    
    \vspace*{1.2cm}
    \Large{Leandro Ferro Luzia}
    
    \vskip 2cm
    \textsc{
    Dissertação apresentada\\[-0.25cm] 
    ao\\[-0.25cm]
    Instituto de Matemática e Estatística\\[-0.25cm]
    da\\[-0.25cm]
    Universidade de São Paulo\\[-0.25cm]
    para\\[-0.25cm]
    obtenção do título\\[-0.25cm]
    de\\[-0.25cm]
    Mestre em Ciências}
    
    \vskip 1.5cm
    Programa: Ciências da Computação\\
    Orientador: Prof. Dr. Francisco C. R. Reverbel
    \vskip 1cm
    
    \vskip 0.5cm
    \normalsize{São Paulo, Abril de 2012}
\end{center}

% ---------------------------------------------------------------------------- %
% Página de rosto (SÓ PARA A VERSÃO DEPOSITADA - ANTES DA DEFESA)
% Resolução CoPGr 5890 (20/12/2010)
\newpage
\thispagestyle{empty}
    \begin{center}
        \vspace*{2.3 cm}
        \textbf{\Large{Uma infraestrutura para desenvolvimento de aplicações distribuídas baseada em minitransações}}\\
        \vspace*{2 cm}
    \end{center}

    \vskip 2cm

    \begin{flushright}
	Esta é a versão original da dissertação elaborada pelo\\
	candidato Leandro Ferro Luzia, tal como \\
	submetida à Comissão Julgadora.
    \end{flushright}

\pagebreak

\pagenumbering{roman}

\chapter*{Agradecimentos}
Texto texto texto texto texto texto texto texto texto texto texto texto texto
texto texto texto texto texto texto texto texto texto texto texto texto texto
texto texto texto texto texto texto texto texto texto texto texto texto texto
texto texto texto texto. Texto opcional.

\chapter*{Resumo}

\noindent LUZIA, L. F. \textbf{Uma infraestrutura para desenvolvimento de aplicações distribuídas baseada em minitransações}. 
2012. 120 f.
Dissertação (Mestrado) - Instituto de Matemática e Estatística,
Universidade de São Paulo, São Paulo, 2012.
\\

Elemento obrigatório, constituído de uma sequência de frases concisas e
objetivas, em forma de texto.  Deve apresentar os objetivos, métodos empregados,
resultados e conclusões.  O resumo deve ser redigido em parágrafo único, conter
no máximo 500 palavras e ser seguido dos termos representativos do conteúdo do
trabalho (palavras-chave). 
Texto texto texto texto texto texto texto texto texto texto texto texto texto
texto texto texto texto texto texto texto texto texto texto texto texto texto
texto texto texto texto texto texto texto texto texto texto texto texto texto
texto texto texto texto texto texto texto texto texto texto texto texto texto
texto texto texto texto texto texto texto texto texto texto texto texto texto
texto texto texto texto texto texto texto texto.
Texto texto texto texto texto texto texto texto texto texto texto texto texto
texto texto texto texto texto texto texto texto texto texto texto texto texto
texto texto texto texto texto texto texto texto texto texto texto texto texto
texto texto texto texto texto texto texto texto texto texto texto texto texto
texto texto.
\\

\noindent \textbf{Palavras-chave:} palavra-chave1, palavra-chave2, palavra-chave3.

\chapter*{Abstract}
\noindent LUZIA, L. F. \textbf{An infrastructure for developing distributed applications based in minitransactions}. 
2010. 120 f.
Dissertação (Mestrado) - Instituto de Matemática e Estatística,
Universidade de São Paulo, São Paulo, 2012.
\\


Elemento obrigatório, elaborado com as mesmas características do resumo em
língua portuguesa. De acordo com o Regimento da Pós- Graduação da USP (Artigo
99), deve ser redigido em inglês para fins de divulgação. 
Text text text text text text text text text text text text text text text text
text text text text text text text text text text text text text text text text
text text text text text text text text text text text text text text text text
text text text text text text text text text text text text.
Text text text text text text text text text text text text text text text text
text text text text text text text text text text text text text text text text
text text text.
\\

\noindent \textbf{Keywords:} keyword1, keyword2, keyword3.

\tableofcontents

\chapter{Lista de Abreviaturas}
\begin{tabular}{ll}
	ACID		& Atomicidade, Consistência, Isolamento e Durabilidade \\
			& (\emph{Atomicity, Consistency, Isolation and Durability})\\
        SGBD	& Sistema Gerenciador de Banco de Dados\\
	2PC		& Efetivação em Duas Fases (\emph{Two-Phase Commit})\\
	TCP/IP	& Conjunto de protocolos de comunicação utilizado na Internet\\
			& (\emph{Transmition Control Protocol} e \emph{Internet Protocol})\\
\end{tabular}

\listoffigures
\listoftables

\mainmatter

% cabeçalho para as páginas de todos os capítulos
\fancyhead[RE,LO]{\thesection}

\singlespacing              % espaçamento simples

\chapter{Introdução}
\label{chap:introducao}
Há diversos motivos para construir uma aplicação de forma distribuída --- compartilhamento de recursos (permitir o acesso a recursos específicos, como impressoras) (TODO coloco essa parte dos recursos??), tolerância a falhas (adicionando mais máquinas ao sistema aumenta a chance de que, a qualquer momento, haja pelo menos uma máquina capaz de responder pelo sistema) (TODO talvez colocar referência a alguma coisa antigona no lamport) e escalabilidade (particionar e distribuir a carga de processamento do sistema permite que este suporte um número crescente de usuários e informações) (TODO ver alguma referencia para escalabilidade tb) são alguns deles.

Um sistema distribuído é composto por diversas máquinas conectadas por uma rede de comunicação \cite{tanenbaum}. Cada processador tem acesso somente ao seu próprio sistema de armazenamento (memória e disco), e a única forma de compartilharem informação é através da troca de dados pela rede de comunicação. Esta troca de dados é feita através da utilização de protocolos de comunicação, como por exemplo os protocolos da família TCP/IP (TODO colocar referência).

Cada máquina no sistema distribuído pode falhar de modo independente, fazendo com que sua parte do estado do sistema fique momentanea ou permanentemente indisponível e que operações no sistema que involvam esta máquina falhem. Se considerarmos o exemplo clássico de um sistema bancário em que as contas dos usuários estão distribuídas entre diversas máquinas e uma solicitação de transferência entre contas que estão em duas máquinas diferentes seja feita, é esperado que esta transferência subtraia uma certa quantia da conta de origem e adicione esta quantia na conta de destino. Se a máquina em que a conta de destino estiver falhar, a quantia subtraída da conta de origem deve ser reposta.

TODO Parei aqui... Muitas aplicações distribuídas utilizam um SGBD (Sistema Gerenciador de Banco de Dados) relacional (como Oracle, MySQL, SQL Server, etc...) como forma de compartilhamento de estado, pois estes sistemas mantém a integridade e a consistência dos dados através da garantia das propriedades ACID (Atomicidade, Consistência, Integridade e Durabilidade) das transações que executam. Essas propriedades facilitam o desenvolvimento da aplicação uma vez que expõem para o desenvolvedor uma interface que:
\begin{enumerate}
	\item Garante que a execução de todas as operações na transação foram efetuadas, ou que nenhuma operação foi feita (Atomicidade)
	\item Garante que os dados estão de acordo com restrições estabelecidas (Consistência)
	\item Garante que duas transações executando de forma concorrente não interfiram no resultado uma da outra (Isolamento)
	\item Garante que as alterações efetuadas por uma transação finalizada com sucesso nunca serão perdidos (Durabilidade)
\end{enumerate}

Há a possibilidade de utilizar mais de um banco de dados para o compartilhamento do estado, e para efetuar o controle transacional entre todos os bancos envolvidos em geral é utilizado o protocolo 2PC (\emph{Two Phase Commit} --- Efetivação em Duas Fases). Este protocolo estabelece que uma máquina seja designada como coordenador da transação, e esta então emite comandos aos participantes da transação, solicitando leituras e escritas. Para concluir a transação, o coordenador solicita que todos os participantes digam se é possível efetivar a transação. Caso algum deles diga que não é possível, a transação é desfeita em todos os participantes. Caso contrário, o coordenador envia um comando de efetivação para todos.

\section{Objetivo}
\label{sec:objetivo}
As minitransações \cite{sinfonia} são uma variação do 2PC, em que os comandos da transação são embutidos nas mensagens do protocolo de efetivação. Assim, na primeira mensagem do protocolo de efetivação, o coordenador envia os comandos da transação. Estes comandos podem ser de comparação (por igualdade), leitura ou escrita. Os comandos de comparação determinam se o participante irá votar para que a transação seja efetivada (caso a comparação resulte em igualdade) ou cancelada (caso a comparação falhe). Os comandos de leitura e escrita só são executados caso o voto seja de efetivação, e o resultado da escrita é armazenado de forma separada para aguardar que o coordenador colete todos os votos e verifique que realmente todos os participantes podem efetivar a transação.

A proposta deste trabalho é implementar uma infraestrutura para sistemas distribuídos que possibilite o compartilhamento de estado entre as máquinas do sistema utilizando minitransações. Ao invés de trocarem mensagens explicitamente, as máquinas verão um repositório que pode crescer de forma a acomodar grandes quantidades de dados e que permite que todas as máquinas tenham sempre acesso a dados consistentes. Assim, esperamos que o desenvolvimento da aplicação distribuída seja mais simples e ajude o desenvolvedor a focar nas necessidades reais da aplicação.

\section{Organização do texto}
\label{sec:organizacao_do_texto}
Este texto é composto, além desta introdução, pelo capítulo \ref{chap:conceitos}, de conceitos e pelo capítulo \ref{chap:conclusoes}, conclusões.

\chapter{Conceitos}
\label{chap:conceitos}

\section{Gerenciamento de Transações}
\label{sec:gerenciamento_de_transacoes}
O gerenciamento de transações tem o objetivo de garantir o isolamento entre comandos concorrentes de diversas aplicações para que os recursos sejam utilizados de forma mais eficiente --- seria possível forçar a execução sequencial dos comandos das aplicações, mas isso desperdiçaria ciclos de processador, enquanto aguarda por leituras ou escritas no disco, e diminuiria a vazão (\emph{throughput}) das respostas.



\section{Minitransações}
\label{sec:minitransacoes}

\chapter{Conclusões}
\label{chap:conclusoes}

% cabeçalho para os apêndices
\renewcommand{\chaptermark}[1]{\markboth{\MakeUppercase{\appendixname\ \thechapter}} {\MakeUppercase{#1}} }
\fancyhead[RE,LO]{}
\appendix

%\include{apendice} 

% ---------------------------------------------------------------------------- %
\backmatter \singlespacing   % espaçamento simples
\bibliographystyle{alpha-ime}% citação bibliográfica alpha
\bibliography{bibliografia}  % associado ao arquivo: 'bibliografia.bib'

% ---------------------------------------------------------------------------- %
% Índice remissivo
%\index{TBP|see{periodicidade região codificante}}
%\index{DSP|see{processamento digital de sinais}}
%\index{STFT|see{transformada de Fourier de tempo reduzido}}
%\index{DFT|see{transformada discreta de Fourier}}
%\index{Fourier!transformada|see{transformada de Fourier}}
%\printindex   % imprime o índice remissivo no documento 

\end{document}
