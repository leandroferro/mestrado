\documentclass[a4paper,12pt]{report}
\usepackage[portuguese,brazilian]{babel}
\usepackage[utf8]{inputenc}
\usepackage[T1]{fontenc}
\usepackage{latexsym}
\title{Uma infraestrutura para o desenvolvimento de sistemas distruídos escaláveis}
\author{Leandro Ferro Luzia}
\date{2012}
\begin{document}
\pagestyle{headings}
\maketitle
\section*{Resumo}
TODO: fazer o resumo mesmo depois de escrever o texto :S

Os sistemas computacionais são utilizados, desde do surgimento dos computadores, para criar e manipular dados de forma sistemática e rápida. Porém, a capacidade de um único computador há muito foi extrapolada , exigindo a distribuição dos dados entre vários computadores ***melhorar isso...

Este trabalho apresenta um paradigma diferente para a construção de sistemas distribuídos escalavéis que propõe que a comunicação entre os componentes do sistema distribuído - feita em geral por troca de mensagens na rede - seja abstraída para a criação e utilização de estruturas de dados no código do sistema. Assim, os desenvolvedores podem passar a se preocupar com a modelagem e a manipulação de estruturas de dados para resolver o problema em questão, enquanto que uma infraestrutura adequada se encarrega de gerenciar estes dados de forma distribuída entre várias máquinas.

Esta infraestrutura provê uma primitiva de programação que permite acessar de forma atômica e condicionalmente modificar dados espalhados entre as máquinas que compõe o sistema. Desta forma, as estruturas de dados podem ser armazenadas de forma distribuída e estar disponíveis para qualquer elemento do sistema de forma consistente.

Será construída esta infraestrutura e algumas aplicações para ilustrar sua utilização. ***completar aqui também
\tableofcontents
\listoftables
\listoffigures
\chapter{Sistemas distribuídos e escalabilidade}
Com a popularização dos computadores, das redes de computadores e, de forma especial, da Internet, os sistemas computacionais tem sido expostos a um número cada vez maior de usuários - tanto humanos quanto computacionais. Desta forma diversos tipos de sistemas (como comércio eletrônico, motores de busca ou redes sociais, por exemplo) precisam suportar uma ampla variação, em termos de números de usuários atendidos e serviços oferecidos, quantidade de dados armazenados e manipulados, requerimentos em relação à taxa de processamento, número de elementos computaticionais que compõem o sistema, distribuição geográfica e tamanho das redes e dispositivos de armazenamento. 

Uma das maneiras de construir sistemas computacionais que atendam aos requisitos apresentados anteriormente é na forma de um sistema distribuído, que é um conjunto de elementos computacionais independentes que passa a impressão, para quem utiliza este sistema, de ser um único elemento computacional \cite{ds-tanenbaum}. Ao construir um sistema computacional de forma distribuída podemos aproveitar de forma mais eficiente os recursos computacionais disponíveis; a disponibilidade do sistema pode ser aumentada pois, com mais elementos computacionais disponíveis para execução, diminui a chance de que o sistema esteja indisponível; e o crescimento do sistema pode ser tratado de forma gradual e sobre demanda. Porém, devido à natureza distribuída destes sistemas problemas podem ser encontrados, que poderiam não existir caso o sistema fosse centralizado, como questões de segurança, falhas de comunicação devido a problemas nas redes de conexão e a chance de falhas, pois agora temos um número maior de componentes no sistema.

A capacidade de um sistema de operar de forma eficiente e com qualidade em relação à variação dos requisitos apresentados anteriormente é chamada de escalabilidade \cite{evaluating-scalability}. A escalabilidade de um sistema pode ser classificada de duas formas \cite{scale-up-vs-down}: escalabilidade {\em vertical} ou {\em horizontal}.

A maneira tradicional de escalar os sistemas era baseada na {\em escalabilidade vertical}, que se refere à capacidade de aumentar a eficiência da execução do sistema ao melhorar o hardware em que o sistema é executado, como por exemplo utilizar processadores com mais núcleos, maior quantidade de memória, discos mais rápidos e confiáveis, etc... Em geral, Sistemas Gerenciadores de Banco de Dados (SGBD's) são escalados utilizando essa abordagem.

Empresas com sistemas voltados à Internet (como Google, eBay, Amazon, Facebook) tem adotado a abordagem da {\em escalabilidade horizontal}, que se refere à capacidade de aumentar a eficiência da execução do sistema ao adicionar mais elementos computacionais ao sistema, permitindo distribuição de carga e aumento da disponibilidade do sistema.
\chapter{Mini-transações}
\chapter{Estruturas de dados como abstração}
\chapter{Implementação}
\chapter{Análise}
\chapter{Trabalhos relacionados}
\chapter{Conclusão}

\begin{thebibliography}{99}

\bibitem{ds-tanenbaum} Tanenbaum, A.S., van Steen, M.: Distributed Systems: Principles and Paradigms, 2nd Edition (2006)

\bibitem{evaluating-scalability} Jogalekar, P., Woodside, M.: Evaluating the scalability of distributed systems. In IEEE Transactions on Parallel and Distributed Systems, Volume 11, Issue 6 (2000)

\bibitem{scale-up-vs-down} Michael, M., Moreira, J.E., Shiloach, D., Wisniewski, R.W.:Scale-up x Scale-out: A Case Study using Nutch/Lucene. In IEEE Parallel and Distributed Processing Symposium (2007)

\bibitem{sinfonia} Aguilera, M.K., Merchant, A., Shah, M., Veitch, A., Karamanolis, C.: Sinfonia: a new paradigm for building scalable distributed systems. Em SOSP, páginas 159-174 (2007)

\end{thebibliography}

\end{document}
