\documentclass[a4paper,12pt]{report}
\usepackage[portuguese,brazilian]{babel}
\usepackage[utf8]{inputenc}
\usepackage[T1]{fontenc}
\usepackage{latexsym}
\title{Uma infraestrutura para o desenvolvimento de sistemas distruídos escaláveis}
\author{Leandro Ferro Luzia}
\date{2012}
\begin{document}
\pagestyle{headings}
\maketitle
\section*{Resumo}
TODO: fazer o resumo mesmo depois de escrever o texto :S

Os sistemas computacionais são utilizados, desde do surgimento dos computadores, para criar e manipular dados de forma sistemática e rápida. Porém, a capacidade de um único computador há muito foi extrapolada , exigindo a distribuição dos dados entre vários computadores ***melhorar isso...

Este trabalho apresenta um paradigma diferente para a construção de sistemas distribuídos escalavéis que propõe que a comunicação entre os componentes do sistema distribuído - feita em geral por troca de mensagens na rede - seja abstraída para a criação e utilização de estruturas de dados no código do sistema. Assim, os desenvolvedores podem passar a se preocupar com a modelagem e a manipulação de estruturas de dados para resolver o problema em questão, enquanto que uma infraestrutura adequada se encarrega de gerenciar estes dados de forma distribuída entre várias máquinas.

Esta infraestrutura provê uma primitiva de programação que permite acessar de forma atômica e condicionalmente modificar dados espalhados entre as máquinas que compõe o sistema. Desta forma, as estruturas de dados podem ser armazenadas de forma distribuída e estar disponíveis para qualquer elemento do sistema de forma consistente.

Será construída esta infraestrutura e algumas aplicações para ilustrar sua utilização. ***completar aqui também
\tableofcontents
\listoftables
\listoffigures
\chapter{Sistemas distribuídos e escalabilidade}
Baseado na definição dada em \cite{ds-tanenbaum}, um sistema distribuído é um conjunto de elementos computacionais independentes que passam a impressão, para quem utiliza o sistema, de ser um único elemento computacional.

Com a popularização das redes de computadores, e de forma especial a Internet, os sistemas computacionais tem sido expostos à um número cada vez maior de usuários - tanto humanos quanto computacionais. Desta forma diversos tipos de sistemas, como comércio eletrônico, busca e redes sociais precisam suportar este número crescente de usuários de forma transparente e 
\chapter{Mini-transações}
\chapter{Estruturas de dados como abstração}
\chapter{Implementação}
\chapter{Análise}
\chapter{Trabalhos relacionados}
\chapter{Conclusão}

\begin{thebibliography}{99}

\bibtem{ds-tanenbaum} Tanenbaum, A.S., van Steen, M.: Distributed Systems: Principles and Paradigms, 2nd Edition (2006)

\bibitem{evaluating-scalability} Jogalekar, P., Woodside, M.: Evaluating the scalability of distributed systems. In IEEE Transactions on Parallel and Distributed Systems, Volume 11, Issue 6 (2000)

\bibitem{sinfonia} Aguilera, M.K., Merchant, A., Shah, M., Veitch, A., Karamanolis, C.: Sinfonia: a new paradigm for building scalable distributed systems. Em SOSP, páginas 159-174 (2007)

\end{thebibliography}

\end{document}
